\documentclass[aip,jcp,preprint,superscriptaddress,nofootinbib]{revtex4-1}
\usepackage{silence}
\WarningFilter{revtex4-1}{Repair the float}
\renewcommand{\thefootnote}{\alph{footnote}}

% Basics
\usepackage[english]{babel}
\usepackage[T1]{fontenc}
\usepackage[utf8]{inputenc}
\usepackage{microtype}

% Maths and physics
\usepackage{amsmath}
\usepackage{amssymb}
\usepackage{textcomp}
\usepackage{textgreek}
\usepackage{physics}
\usepackage{upgreek}
\usepackage{mathtools}
\usepackage{bbm}
% \usepackage{bbold}
\usepackage{bm}
\usepackage{mathrsfs}
\usepackage{nicefrac}

% Graphics and layout
\usepackage{graphicx}
\usepackage{xcolor}
\usepackage{float}
\usepackage{booktabs}
\usepackage{enumitem}
\usepackage[colorlinks]{hyperref}
\usepackage{braket}
\usepackage{array}
%\usepackage{titlesec}
\usepackage{siunitx}
	\sisetup{group-digits=integer}
\usepackage{threeparttable}
%\usepackage{pifont}
\usepackage{lipsum}

\frenchspacing
% Misc.
\newcommand{\elm}[3]{\langle #1 \vert #2 \vert #3 \rangle}
\newcommand{\nn}{\nonumber\\}
\newcommand{\bigbraket}[2]{\Big\langle #1 \mspace{2mu} \Big\vert \mspace{1.5mu} #2 \Big\rangle}


% Creating operators
\newcommand{\crea}[1]{c^{\dagger}_{#1}}
\newcommand{\creasuper}[2]{c^{\dagger}_{#1^{\smash{#2}}}}
\newcommand{\creadot}[1]{\dot{c}^{\dagger}_{#1}}
\newcommand{\creavar}[1]{\delta c^{\dagger}_{#1}}
\newcommand{\creasuperdot}[2]{\dot{c}^{\dagger}_{#1^{\smash{#2}}}}

% Annihilation operators
\newcommand{\anni}[1]{\tilde{c}_{#1}}
\newcommand{\annisuper}[2]{\tilde{c}_{{#1}^{#2}}}
\newcommand{\annidot}[1]{\dot{\tilde{c}}_{#1}}
\newcommand{\annivar}[1]{\delta\tilde{c}_{#1}}

\newcommand{\creaprim}[1]{a^{\dagger}_{#1}}
\newcommand{\anniprim}[1]{\tilde{a}_{#1}}

% Dual and primal active basis functions
\newcommand{\phibra}[1]{\tilde{\varphi}_{#1}}
\newcommand{\phiket}[1]{\varphi_{#1}}
\newcommand{\chibra}[1]{\tilde{\chi}_{#1}}
\newcommand{\chiket}[1]{\chi_{#1}}
\newcommand{\phibradot}[1]{\dot{\tilde{\varphi}}_{#1}}
\newcommand{\phiketdot}[1]{\dot{\varphi}_{#1}}
\newcommand{\phibravar}[1]{\delta\tilde{\varphi}_{#1}}
\newcommand{\phiketvar}[1]{\delta\varphi_{#1}}

% Shift operators
\newcommand{\Eplain}[2]{E_{{#1} {#2}}}




% % Struts for extra spacing in tables/matrices
% \newcommand{\topstrut}[1]{\rule{0pt}{#1}}
% \newcommand{\bottomstrut}[1]{\rule[#1]{0pt}{0pt}}
% \newcommand{\bothstrut}[2]{\rule[#1]{0pt}{#2}}
% \newcommand\Tstrut{\rule{0pt}{2.4ex}}
% \newcommand\Bstrut{\rule[-0.9ex]{0pt}{0pt}}
% \newcommand\TTstrut{\rule{0pt}{3.0ex}}
% \newcommand\BBstrut{\rule[-1.5ex]{0pt}{0pt}}

% % Custom typographical commands
% \newcommand{\mrm}[1]{\mathrm{#1}}
% \newcommand{\mbf}[1]{\mathbf{#1}}
% \newcommand{\nn}{\nonumber\\}
% \newcommand{\trans}{\mspace{-3mu}\mathsf{T}}
% \newcommand{\subA}{\mathrm{\scriptscriptstyle A}}
% \newcommand{\subS}{\mathrm{\scriptscriptstyle S}}
% \newcommand{\bigbraket}[2]{\Big\langle #1 \mspace{2mu} \Big\vert \mspace{1.5mu} #2 \Big\rangle}
% \newcommand{\sss}[1]{\scriptscriptstyle #1}
% \newcommand{\2}{\mspace{-2mu}}


% % raised chi
% \DeclareRobustCommand{\raisedchi}{{\mathpalette\irchi\relax}}
% \newcommand{\irchi}[2]{\raisebox{\depth}{$#1\chi$}}

% % raised rho
% \DeclareRobustCommand{\raisedrho}{{\mathpalette\irrho\relax}}
% \newcommand{\irrho}[2]{\raisebox{1pt}{$#1\rho$}}

% % dcases environment with centering
% \makeatletter
% \renewenvironment{dcases}[1][l]{\matrix@check\dcases\env@dcases{#1}}{\endarray\right.}
% \def\env@dcases#1{%
%   \let\@ifnextchar\new@ifnextchar
%   \left\lbrace\def\arraystretch{1.2}%
%   \array{@{}#1@{\quad}l@{}}}
% \makeatother

% % custom bar
% \newcommand*\xbar[1]{%
%   \hbox{%
%   \kern0.3em%
%     \vbox{%
%       \hrule height 0.5pt % The actual bar
%       \kern0.3ex%         % Distance between bar and symbol
%       \hbox{%
%         \kern-0.3em%      % Shortening on the left side
%         \ensuremath{#1}%
%         \kern-0.2em%      % Shortening on the right side
%       }%
%     }%
%     \kern0.2em%
%   }%
% } 

% \DeclareFontFamily{OT1}{pzc}{}
% \DeclareFontShape{OT1}{pzc}{m}{it}{<-> s * [1.10] pzcmi7t}{}
% \DeclareMathAlphabet{\mathpzc}{OT1}{pzc}{m}{it}

% % Shift operators and similar
% \newcommand{\Eplain}[3]{E^{#1}_{{#2} {#3}}}
% \newcommand{\Etilde}[3]{\tilde{E}^{#1}_{{#2} {#3}}}
% \newcommand{\Eplainprime}[3]{E^{#1'}_{{#2} {#3}}}
% \newcommand{\Etildeprime}[3]{\tilde{E}^{#1'}_{{#2} {#3}}}

% \newcommand{\Dplain}[3]{D^{#1}_{{#2} {#3}}}
% \newcommand{\Dtilde}[3]{\widetilde{D}^{#1}_{{#2} {#3}}}
% \newcommand{\Dplainprime}[3]{D^{#1'}_{{#2} {#3}}}
% \newcommand{\Dtildeprime}[3]{\widetilde{D}^{#1'}_{{#2} {#3}}}

% % Kappa matrix elements
% \newcommand{\kappaplain}[3]{\kappa^{#1}_{{#2} {#3}}}
% \newcommand{\kappaplaindot}[3]{\dot{\kappa}^{#1}_{{#2} {#3}}}
% \newcommand{\kappaplainprime}[3]{\kappa^{#1'}_{{#2} {#3}}}

% % G matrix elements
% \newcommand{\gplain}[3]{g^{#1}_{{#2} {#3}}}
% \newcommand{\gplainprime}[3]{g^{#1'}_{{#2} {#3}}}

% % Mean-field matrices
% \newcommand{\FFcheckplain}[1]{\check{\mbf{F}}^{#1}}
% \newcommand{\FFcheckprime}[1]{\check{\mbf{F}}^{\prime#1}}
% \newcommand{\FFcheckplaindagger}[1]{\check{\mbf{F}}^{#1\dagger}}
% \newcommand{\FFcheckprimedagger}[1]{\check{\mbf{F}}^{\prime#1\dagger}}
% \newcommand{\FFtildeplain}[1]{\tilde{\mbf{F}}^{#1}}
% \newcommand{\FFtildeprime}[1]{\tilde{\mbf{F}}^{\prime#1}}
% \newcommand{\FFtildeplaindagger}[1]{\tilde{\mbf{F}}^{#1\dagger}}
% \newcommand{\FFtildeprimedagger}[1]{\tilde{\mbf{F}}^{\prime#1\dagger}}

% % Mean-field matrix elements
% \newcommand{\Fcheckplain}[3]{\check{F}^{#1}_{{#2} {#3}}}
% \newcommand{\Fcheckplainconj}[3]{\check{F}^{#1*}_{{#2} {#3}}}
% \newcommand{\Fcheckprime}[3]{\check{F}^{\prime#1}_{{#2} {#3}}}
% \newcommand{\Ftildeplain}[3]{\tilde{F}^{#1}_{{#2} {#3}}}
% \newcommand{\Ftildeprime}[3]{\tilde{F}^{\prime#1}_{{#2} {#3}}}
% \newcommand{\Fplain}[3]{F^{#1}_{{#2} {#3}}}
% \newcommand{\fplain}[3]{f_{(#1  {#2} {#3})}}
% \newcommand{\fbarplain}[3]{\bar{f}_{(#1  {#2} {#3})}}
% \newcommand{\fbarprime}[3]{\bar{f}^{\prime}_{(#1  {#2} {#3})}}
% \newcommand{\fbarprimeconj}[3]{\bar{f}^{\prime*}_{(#1  {#2} {#3})}}

% % Density elements
% \newcommand{\rrho}[3]{\rho^{#1}_{#2 #3}}
% \newcommand{\rrhodot}[3]{\dot{\rho}^{#1}_{#2 #3}}
% \newcommand{\rrhoconj}[3]{\rho^{#1*}_{#2 #3}}

% % Z matrix elements
% \newcommand{\Zplain}[3]{Z^{#1}_{#2^{#1} #3^{#1}}}
% \newcommand{\Zplainconj}[3]{Z^{#1*}_{#2^{#1} #3^{#1}}}


% % D matrix elements
% \newcommand{\dplain}[5]{d^{#1}_{({#2} {#3}) ({#4} {#5})}}
% \newcommand{\dtilde}[5]{\tilde{d}^{#1}_{({#2} {#3}) ({#4} {#5})}}
% \newcommand{\dplainprime}[5]{d^{#1'}_{ ({#2} {#3}) ({#4} {#5}) } }
% \newcommand{\dfullplain}[6]{d_{ ({#1} \, {#2}  {#3}) ({#4'} {#5} {#6}) } }

% % C matrix elements
% \newcommand{\Cprime}[6]{C^{\prime}_{ ({#1}  {#2}^{#1}  {#3}^{#1}) ({#4'} {#5}^{#4^{\smash{\prime}}} \! {#6}^{#4^{\smash{\prime}}}) } }
% \newcommand{\Cprimeconj}[6]{C^{\prime*}_{ ({#1}  {#2}^{#1}  {#3}^{#1}) ({#4'} {#5}^{#4^{\smash{\prime}}} \! {#6}^{#4^{\smash{\prime}}}) } }
% \newcommand{\Cplain}[6]{C_{ ({#1} {#2}  {#3}) ({#4'} {#5} {#6}) } }
% \newcommand{\Cbarplain}[6]{\bar{C}_{ ({#1}  {#2}^{#1}  {#3}^{#1}) ({#4'} {#5}^{#4^{\smash{\prime}}} \! {#6}^{#4^{\smash{\prime}}}) } }
% \newcommand{\Cbarprime}[6]{\bar{C}^{\prime}_{ ({#1}  {#2}^{#1}  {#3}^{#1}) ({#4'} {#5}^{#4^{\smash{\prime}}} \! {#6}^{#4^{\smash{\prime}}}) } }
% \newcommand{\Cbarconj}[6]{\bar{C}^{*}_{ ({#1}  {#2}^{#1}  {#3}^{#1}) ({#4'} {#5}^{#4^{\smash{\prime}}} \! {#6}^{#4^{\smash{\prime}}}) } }
% \newcommand{\Cbarprimeconj}[6]{\bar{C}^{\prime*}_{ ({#1}  {#2}^{#1}  {#3}^{#1}) ({#4'} {#5}^{#4^{\smash{\prime}}} \! {#6}^{#4^{\smash{\prime}}}) } }


% % A matrix elements
% \newcommand{\Aprime}[4]{A_{ {#1} ({#2'} {#3} {#4}) } }
% \newcommand{\Aplain}[4]{A_{ {#1} ({#2}  {#3} {#4}) } }
% \newcommand{\Atildeprime}[4]{\tilde{A}_{ {#1} ({#2'} {#3}^{#2^{\smash{\prime}}} \! {#4}^{#2^{\smash{\prime}}}) } }
% \newcommand{\Atildeplain}[4]{\tilde{A}_{ {#1} ({#2} \, {#3}^{#2} {#4}^{#2}) } }

% % S matrix elements
% \newcommand{\Splain}[6]{S_{ ({#1} \, {#2}^{#1}  {#3}^{#1}) ({#4'} {#5}^{#4^{\smash{\prime}}} \! {#6}^{#4^{\smash{\prime}}}) } }

% % Creating/annihilation operators
% \newcommand{\aplain}[2]{a^{#1\mspace{-0.5mu}\raisebox{0.3ex}{$\scriptstyle\dagger$}}_{\mspace{-0mu}#2}}
% \newcommand{\aprime}[2]{a^{#1^{\smash{\prime}}\mspace{-0.5mu}\raisebox{0.6ex}{$\scriptstyle\dagger$}}_{\mspace{-0mu}#2}}
% \newcommand{\bplain}[2]{b^{#1}_{\mspace{-1mu}#2}}
% \newcommand{\bprime}[2]{b^{#1^{\smash{\prime}}}_{\mspace{-1mu}#2}}
% \newcommand{\crea}[2]{a^{#1\mspace{-0.5mu}\raisebox{0.3ex}{$\scriptstyle\dagger$}}_{\mspace{-0mu}#2}}
% \newcommand{\creatilde}[2]{\tilde{a}^{#1\mspace{-0.5mu}\raisebox{0.3ex}{$\scriptstyle\dagger$}}_{\mspace{-0mu}#2}}
% \newcommand{\anni}[2]{a^{#1}_{\mspace{-0mu}#2}}
% \newcommand{\annitilde}[2]{\tilde{a}^{#1}_{\mspace{-0mu}#2}}
% \newcommand{\annitildeprime}[2]{\tilde{a}^{#1^{\smash{\prime}}}_{\mspace{-0mu}#2}}

% % Excitation/deexcitation operators
% \newcommand{\tauplaindagger}[2]{\tau^{#1\mspace{-0.5mu}\raisebox{0.3ex}{$\scriptstyle\ddagger$}}_{#2}}
% \newcommand{\tauplain}[2]{\tau^{#1}_{#2}}

% % Modal matrix elements
% \newcommand{\uplain}[3]{U^{#1}_{#2 #3}}
% \newcommand{\uplaindot}[3]{\dot{U}^{#1}_{#2 #3}}
% \newcommand{\wplain}[3]{W^{#1}_{#2 #3}}
% \newcommand{\wplaindot}[3]{\dot{W}^{#1}_{#2 #3}}
% \newcommand{\vplain}[3]{V^{#1}_{\mspace{-2mu} #2 #3}}
% \newcommand{\vplainconj}[3]{V^{#1*}_{\mspace{-2mu} #2 #3}}
% \newcommand{\vplaindot}[3]{\dot{V}^{#1}_{\mspace{-2mu} #2 #3}}
% \newcommand{\vplaindotconj}[3]{\dot{V}^{#1*}_{\mspace{-2mu} #2 #3}}
\usepackage{acro}

\DeclareAcronym{mctdh}{
   short = MCTDH ,
   long = multiconfiguration time-dependent Hartree ,
}

\DeclareAcronym{nomctdh}{
   short = NOMCTDH ,
   long = non-orthogonal \ac{mctdh} ,
}

\DeclareAcronym{gmctdh}{
   short = G-MCTDH ,
   long = Gaussian-based \ac{mctdh} ,
}

\DeclareAcronym{mlgmctdh}{
   short = ML-GMCTDH ,
   long = multilayer Gaussian-based \ac{mctdh} ,
}

\DeclareAcronym{mlmctdh}{
   short = ML-MCTDH ,
   long = multilayer \ac{mctdh} ,
}

\DeclareAcronym{mpsmctdh}{
   short = MPS-MCTDH ,
   long = matrix product state \ac{mctdh} ,
}

\DeclareAcronym{vmcg}{
   short = vMCG ,
   long = variational multiconfiguration Gaussian ,
}

\DeclareAcronym{ms}{
   short = MS ,
   long = multiple spawning ,
}

\DeclareAcronym{ccs}{
   short = CCS ,
   long = coupled coherent states ,
}

\DeclareAcronym{mctdhn}{
   short = MCTDH[\textit{n}] ,
   long = systematically truncated multiconfiguration time-dependent Hartree ,
}

\DeclareAcronym{mrmctdhn}{
  short = MR-MCTDH[\textit{n}] ,
  long = multi-reference truncated multiconfiguration time-dependent Hartree ,
}

\DeclareAcronym{tdh}{
   short = TDH ,
   long = time-dependent Hartree ,
}

\DeclareAcronym{dmrg}{
   short = DMRG ,
   long = density matrix renormalization group,
}

\DeclareAcronym{tddmrg}{
   short = TD-DMRG ,
   long = time-dependent density matrix renormalization group,
}

\DeclareAcronym{scf}{
   short = SCF ,
   long = self-consistent field ,
}

\DeclareAcronym{casscf}{
   short = CASSCF ,
   long = complete active space self-consistent field ,
}

\DeclareAcronym{tdcasscf}{
   short = TD-CASSCF ,
   long = time-dependent \acl{casscf} ,
}

\DeclareAcronym{gasscf}{
   short = CASSCF ,
   long = generalized active space self-consistent field ,
}

\DeclareAcronym{tdgasscf}{
   short = TD-GASSCF ,
   long = time-dependent \acl{gasscf} ,
}

\DeclareAcronym{rasscf}{
   short = RASSCF ,
   long = restricted active space self-consistent field ,
}

\DeclareAcronym{tdrasscf}{
   short = TD-RASSCF ,
   long = time-dependent \acl{rasscf} ,
}

\DeclareAcronym{ormas}{
   short = ORMAS ,
   long = occupation-restricted multiple active space ,
}
\DeclareAcronym{tdormas}{
   short = TD-ORMAS ,
   long = time-dependent \acl{ormas} ,
   %long = time-dependent occupation-restricted multiple active space ,
}

\DeclareAcronym{mctdhf}{
   short = MCTDHF ,
   long = multiconfiguration time-dependent Hartree-Fock ,
}

\DeclareAcronym{occ}{
   short = OCC ,
   long = orbital-optimized coupled cluster ,
}

\DeclareAcronym{tdocc}{
   short = TD-OCC ,
   long = time-dependent \acl{occ} ,
}

\DeclareAcronym{nocc}{
   short = NOCC ,
   long = non-orthogonal orbital-optimized coupled cluster ,
}

\DeclareAcronym{oatdcc}{
   short = OATDCC ,
   long = orbital-adaptive time-dependent coupled cluster ,
}

\DeclareAcronym{fci}{
   short = FCI ,
   long = full configuration interaction ,
}

\DeclareAcronym{cud}{
   short = CUD ,
   long = closed under de-exciation ,
}

\DeclareAcronym{fsmr}{
   short = FSMR ,
   long = full-space matrix representation ,
}

\DeclareAcronym{hh}{
   short = HH ,
   long = H\'enon-Heiles ,
}

\DeclareAcronym{ho}{
   short = HO ,
   long = harmonic oscillator ,
}

\DeclareAcronym{dop853}{
   short = DOP853 ,
   long = Dormand-Prince 8{(5,3)} ,
}

\DeclareAcronym{sm}{
   short = SM ,
   long = supplementary material ,
}

\DeclareAcronym{vscf}{
   short = VSCF ,
   long = vibrational self-consistent field ,
}

\DeclareAcronym{eom}{
   short = EOM ,
   long = equation of motion ,
   short-plural-form = EOMs ,
   long-plural-form = equations of motion ,
}

\DeclareAcronym{tdvp}{
   short = TDVP ,
   long = time-dependent variational principle
}

\DeclareAcronym{tdse}{
   short = TDSE ,
   long = time-dependent Schr{\"o}dinger equation ,
}

\DeclareAcronym{cc}{
   short = CC ,
   long = coupled cluster ,
}

\DeclareAcronym{vcc}{
   short = VCC ,
   long = vibrational coupled cluster ,
}

\DeclareAcronym{tdvcc}{
   short = TDVCC ,
   long = time-dependent vibrational coupled cluster ,
}

\DeclareAcronym{tdvci}{
   short = TDVCI ,
   long = time-dependent vibrational configuration interaction ,
}

\DeclareAcronym{vci}{
   short = VCI ,
   long = vibrational configuration interaction ,
}

\DeclareAcronym{ci}{
   short = CI ,
   long = configuration interaction ,
}

\DeclareAcronym{tdci}{
   short = CI ,
   long = time-dependent \acl{ci} ,
}

\DeclareAcronym{sq}{
   short = SQ ,
   long = second quantization ,
}

\DeclareAcronym{fq}{
   short = FQ ,
   long = first quantization ,
}

\DeclareAcronym{mc}{
   short = MC ,
   long = mode combination ,
}

\DeclareAcronym{mcr}{
   short = MCR ,
   long = mode combination range ,
   long-plural = s ,
}

\DeclareAcronym{pes}{
   short = PES ,
   long = potential energy surface
}

\DeclareAcronym{svd}{
   short = SVD ,
   long = singular value decomposition ,
}
\DeclareAcronym{adga}{
   short = ADGA ,
   long = adaptive density-guided approach ,
}

\DeclareAcronym{rhs}{
   short = RHS ,
   long = right-hand side ,
}

\DeclareAcronym{lhs}{
   short = LHS ,
   long = left-hand side ,
}

\DeclareAcronym{ivr}{
   short = IVR ,
   long = intramolecular vibrational energy redistribution ,
}

\DeclareAcronym{fft}{
   short = FFT ,
   long = fast Fourier transform ,
}

\DeclareAcronym{spf}{
   short = SPF ,
   long = single-particle function ,
}


\DeclareAcronym{lls}{
   short = LLS ,
   long = linear least squares ,
}

\DeclareAcronym{itnamo}{
   short = ItNaMo ,
   long = iterative natural modal ,
}

\DeclareAcronym{hf}{
   short = HF ,
   long = Hartree-Fock ,
}

\DeclareAcronym{mcscf}{
   short = MCSCF ,
   long = multi-configurational self-consistent field ,
}

\DeclareAcronym{sop}{
   short = SOP ,
   long = sum-of-products ,
}
\DeclareAcronym{midascpp}{
   short = MidasCpp ,
   long = Molecular Interactions{,} Dynamics And Simulations Chemistry Program Package ,
   tag = abbrev ,
}

\DeclareAcronym{mpi}{
   short = MPI ,
   long = message passing interface ,
}

\DeclareAcronym{ode}{
   short = ODE ,
   long  = ordinary differential equation ,
   short-plural = s ,
   long-plural = s ,
   short-indefinite = an ,
   long-indefinite = an ,
   tag = abbrev ,
}

\DeclareAcronym{bch}{
   short = BCH ,
   long = Baker-Campbell-Hausdorff ,
}

\DeclareAcronym{sr}{
   short = SR ,
   long = single-reference ,
}
\DeclareAcronym{mr}{
   short = MR ,
   long = multi-reference ,
}

\DeclareAcronym{dof}{
   short = DOF ,
   long = degree of freedom ,
   short-plural-form = DOFs ,
   long-plural-form = degrees of freedom ,
}

\DeclareAcronym{hp}{
   short = HP ,
   long = Hartree product ,
}

\DeclareAcronym{tdbvp}{
   short = TDBVP ,
   long  = time-dependent bivariational principle ,
   short-plural = s ,
   long-plural = s ,
   short-indefinite = a ,
   long-indefinite = a ,
   tag = abbrev ,
}

\DeclareAcronym{dfvp}{
   short = DFVP ,
   long  = Dirac-Frenkel variational principle ,
}

\DeclareAcronym{ele}{
   short = ELE ,
   long  = Euler-Lagrange equation ,
   short-plural = s ,
   long-plural = s ,
   tag = abbrev ,
}

\DeclareAcronym{mrcc}{
   short = MRCC ,
   long = multi-reference coupled cluster ,
}

\DeclareAcronym{tdfvci}{
   short = TDFVCI ,
   long = time-dependent full vibrational configuration interaction ,
}

\DeclareAcronym{tdfci}{
   short = TDFCI ,
   long = time-dependent full configuration interaction ,
}

\DeclareAcronym{tdevcc}{
   short = TDEVCC ,
   long  = time-dependent extended vibrational coupled cluster ,
   short-plural = s ,
   long-plural = s ,
   short-indefinite = a ,
   long-indefinite = a ,
   tag = abbrev ,
}

\DeclareAcronym{holc}{
   short = HOLC ,
   long = hybrid optimized and localized vibrational coordinate ,
}

\DeclareAcronym{acf}{
   short = ACF ,
   long = autocorrelation function ,
}

\DeclareAcronym{fwhm}{
   short = FWHM ,
   long  = full width at half maximum ,
   short-plural = s ,
   long-plural = full widths at half maxima ,
   short-indefinite = an ,
   long-indefinite = a ,
   tag = abbrev ,
}

\DeclareAcronym{tdmvcc}{
   short = TDMVCC ,
   long = time-dependent modal vibrational coupled cluster ,
}

\DeclareAcronym{otdmvcc}{
   short = oTDMVCC ,
   long = orthogonal time-dependent modal vibrational coupled cluster ,
}

\DeclareAcronym{midas}{
   short = MidasCpp ,
   long = Molecular Interactions{,} Dynamics and Simulations Chemistry Program Package ,
}


\newcommand{\au}{Department of Chemistry, Aarhus University, Langelandsgade 140, 8000 Aarhus C, Denmark}


\begin{document}

\title{Relating orbital and parameter equations}
\author{Mads Greisen Højlund}
\email{madsgh@chem.au.dk}
\affiliation{\au}

\date{\today}
\maketitle

\tableofcontents
\newpage



\section{Notation} \label{sec:notation}

The Hamiltonian is written using physicist's notation for the integrals (which are not anti-symmetrized):
\begin{align}
    H = \sum_{pq} h_{pq} \crea{p} \anni{q} + \frac{1}{2} \sum_{pqrs} u_{pqrs} \crea{p} \crea{q} \anni{s} \anni{r}
\end{align}
\begin{align}
    h_{pq} 
    = \elm{\phibra{p}}{h}{\phiket{q}} 
    \equiv \int \phibra{p}(\mathbf{x}) h(\mathbf{x}) \phiket{q}(\mathbf{x}) \dd{\mathbf{x}}
\end{align}
\begin{align}
    u_{pqrs} 
    = \elm{\phibra{p} \phibra{q}}{u}{\phiket{r} \phiket{s}}
    \equiv \iint \phibra{p}(\mathbf{x}) \phibra{q}(\mathbf{y}) u(\mathbf{x}, \mathbf{y}) \phiket{r}(\mathbf{x}) \phiket{s}(\mathbf{y}) \dd{\mathbf{x}} \dd{\mathbf{y}}
\end{align}
Density matrices are defined as
\begin{align}
    \rho_{qp}   &= \elm{\tilde{\Psi}}{ \crea{p} \anni{q} }{\Psi}, \\
    \rho_{rspq} &= \elm{\tilde{\Psi}}{ \crea{p} \crea{q} \anni{s} \anni{r} }{\Psi}.
\end{align}
The two-electron integrals and densities have the following useful permutation symmetries:
\begin{align}
    u_{pqrs}    &= u_{qpsr}, \\
    \rho_{rspq} &= \rho_{srqp}.
\end{align}


\section{Converting orbital equations to parameter equations}
The orbital equations are formally identical for all orbital-adaptive methods.
Following Kvaal\cite{kvaalInitioQuantumDynamics2012} and Sato\cite{satoCommunicationTimedependentOptimized2018}, 
they take the appearance
\begin{align}
    i \ket{\phiketdot{p}} 
    = i \sum_{q} \ket{\phiket{q}} \eta_{qp}
    + Q \Big[
    h \ket{\phiket{p}} + \sum_{oqrs} W_{rs} \ket{\phiket{q}} P_{qsor} \, (\mathbf{D}^{-1})_{op}
    \Big], \label{eq:phiket_dot} \\
    i \bra{\phibradot{p}} 
    = -i \sum_{q} \eta_{pq} \bra{\phibra{q}} -
    \Big[
    \bra{\phibra{p}} h + \sum_{oqrs} (\mathbf{D}^{-1})_{po} \, P_{osqr}  \bra{\phibra{q}} W_{rs} 
    \Big] Q. \label{eq:phibra_dot}
\end{align}
The mean-field operator\cite{miyagiTimedependentRestrictedactivespaceSelfconsistentfield2013,satoTimedependentCompleteactivespaceSelfconsistentfield2013}
$W_{rs}$ is a multiplicative/local operator given by
\begin{align}
    W_{rs} (\mathbf{x}) = \int \phibra{r} (\mathbf{y}) u(\mathbf{x}, \mathbf{y}) \phiket{s} (\mathbf{y}) \dd{\mathbf{y}}.
\end{align}
Kvaal\cite{kvaalInitioQuantumDynamics2012} uses a slightly different notation, but the meaning should be the same. 
The secondary-space projector is given by
\begin{align}
    Q = 1 - \sum_{p} \ket{\phiket{p}} \! \bra{\phibra{p}}.
\end{align}
The definitions of the densities depend on whether biorthogonal\cite{kvaalInitioQuantumDynamics2012}
or orthogonal\cite{satoTimedependentCompleteactivespaceSelfconsistentfield2013} orbitals are used.
In the biorthogonal case, the plain one- and two-electron densities are used:
\begin{align}
    D_{qp}   &= \rho_{qp}, \\
    P_{rspq} &= \rho_{rspq}.
\end{align}
In the orthogonal case, the densities are Hermitianized according to
\begin{align}
    D_{qp}   &= \tfrac{1}{2} ( \rho_{qp}   + \rho_{pq}^*   ), \\
    P_{rspq} &= \tfrac{1}{2} ( \rho_{rspq} + \rho_{pqrs}^* ).
\end{align}
Having on orthogonal basis of course means $\bra{\phibra{p}} = \bra{\phiket{p}}$,
which implies $Q = Q^{\dagger}$ and $W_{rs}^* = W_{sr}$. Using these relations
and the Hermitianized densities one easily confirms that Eqs.~\eqref{eq:phiket_dot}
and \eqref{eq:phibra_dot} are simply each other's adjoint,
when the orbitals are orthogonal.

We now introduce an underlying/primitive basis that is assumed to be biorthonormal.
The time-dependent orbitals are given in terms of this basis as
\begin{align}
    \ket{\phiket{p}} = \sum_{\alpha} \ket{\chiket{\alpha}} C_{\alpha p},         \label{eq:phi_ket_expansion} \\
    \bra{\phibra{p}} = \sum_{\alpha} \tilde{C}_{p \alpha} \bra{\chibra{\alpha}}, \label{eq:phi_bra_expansion}
\end{align}
with time-dependent coefficients. The primitive basis induces the following identity in
the one-particle space:
\begin{align}
    1 = \sum_{\beta} \ket{\chiket{\beta}} \! \bra{\chibra{\beta}} 
\end{align}
Inserting this identity into Eq.~\eqref{eq:phiket_dot} and projecting onto $\bra{\chibra{\alpha}}$ yields
\begin{multline}
    i \braket{\chibra{\alpha} | \phiketdot{p}} 
    = i \sum_{q} \braket{\chibra{\alpha} | \phiket{q}} \eta_{qp}
    + \sum_{\beta} \elm{\chibra{\alpha}}{Q}{\chiket{\beta}} \Big[
    \elm{\chibra{\beta}}{h}{\phiket{p}}
    + \sum_{oqrs}  \elm{\chibra{\beta}}{W_{rs}}{\phiket{q}} P_{qsor} \, (\mathbf{D}^{-1})_{op}
    \Big].
\end{multline}
Introducing matrix notation, this reads
\begin{align} \label{eq:C_dot}
    i \dot{\mathbf{C}} 
    &= i \mathbf{C} \bm{\eta} + 
    \mathbf{Q}  \big( \check{\mathbf{H}} + \check{\mathbf{F}} \mathbf{D}^{-1} \big) \nn
    &= i \mathbf{C} \bm{\eta} + 
    \mathbf{Q}  \big( \check{\mathbf{H}} \mathbf{D} + \check{\mathbf{F}} \big) \mathbf{D}^{-1} .
\end{align}
The secondary-space projector has matrix elements
\begin{align}
    Q_{\alpha \beta} 
    &= \elm{\chibra{\alpha}}{Q}{\chiket{\beta}} \nn
    &= \elm{\chibra{\alpha}}{(1 - P)}{\chiket{\beta}} \nn
    &= \braket{\chibra{\alpha} | \chiket{\beta}} - \sum_{p} \braket{\chibra{\alpha} | \phiket{p}} \braket{\phibra{p} | \chiket{\beta}}
\end{align}
or simply
\begin{align}
    \mathbf{Q} = \mathbf{1} - \mathbf{C} \tilde{\mathbf{C}}.
\end{align}
The matrices $\check{\mathbf{H}}$ and $\check{\mathbf{F}}$ contain
half-transformed one-electron integrals,
\begin{gather}
    \check{H}_{\beta p} = \elm{\chibra{\beta}}{h}{\phiket{p}} = h_{\beta p},
\end{gather}
and half-transformed mean-field elements,
\begin{align}
    \check{F}_{\beta o}
    &= \sum_{qrs}  \elm{\chibra{\beta}}{W_{rs}}{\phiket{q}} P_{qsor} \nn
    &= \sum_{qrs}
    \int \chibra{\beta} (\mathbf{x}) \left[ 
    \int \phibra{r}(\mathbf{y}) u(\mathbf{x}, \mathbf{y}) \phiket{s}(\mathbf{y})  \dd{\mathbf{y}} 
    \right] \phiket{q} (\mathbf{x}) \dd{\mathbf{x}}
    P_{qsor} \nn
    &= \sum_{qrs}
    \iint \chibra{\beta} (\mathbf{x}) \phibra{r}(\mathbf{y}) u(\mathbf{x}, \mathbf{y}) \phiket{q} (\mathbf{x}) \phiket{s}(\mathbf{y}) \dd{\mathbf{x}} \dd{\mathbf{y}} 
    P_{qsor} \nn
    &= \sum_{qrs}  u_{\beta r q s} P_{qsor}. \label{eq:F_check_from_u_and_rho}
\end{align}
Using similar steps, Eq.~\eqref{eq:phibra_dot} leads to
\begin{align} \label{eq:C_tilde_dot}
    i \dot{\tilde{\mathbf{C}}} 
    &= -i  \bm{\eta} \tilde{\mathbf{C}} - 
    \big( \check{\mathbf{H}}' + \mathbf{D}^{-1} \check{\mathbf{F}}' \big) \mathbf{Q} \nn
    &= -i  \bm{\eta} \tilde{\mathbf{C}} - 
    \mathbf{D}^{-1} \big( \mathbf{D} \check{\mathbf{H}}' + \check{\mathbf{F}}' \big) \mathbf{Q}
\end{align}
with
\begin{gather}
    \check{H}'_{p \beta} = \elm{\phibra{p}}{h}{\chiket{\beta}} = h_{p \beta},  \\
    \check{F}'_{o \beta} = \sum_{qrs} P_{osqr} u_{qr\beta s}.\label{eq:F_check_prime_from_u_and_rho}
\end{gather}

\section{Relation to other work}
In Refs.~\citenum{madsenTimedependentVibrationalCoupled2020} and \citenum{hojlundBivariationalTimedependentWave2022},
the parameter equations were derived (for the biorthogonal case) by assuming the expansions in
Eqs.~\eqref{eq:phi_ket_expansion} and \eqref{eq:phi_bra_expansion} from the outset.
Those derivations also apply to the electronic structure problem
(this is an explicit point in Ref.\citenum{hojlundBivariationalTimedependentWave2022}), so we
should check that they agree with Eqs.~\eqref{eq:C_dot} and \eqref{eq:C_tilde_dot}.
Ignoring notational differences, we need to check that
\begin{align}
    \big[ \check{\mathbf{H}} \mathbf{D} + \check{\mathbf{F}} \big]_{\bar{\alpha} \bar{p}}
    &= \elm{\tilde{\Psi}}{ \crea{\bar{p}} [\anniprim{\bar{\alpha}}, H] }{\Psi},  \label{eq:F_check_correspondence} \\
    \big[ \mathbf{D} \check{\mathbf{H}}' + \check{\mathbf{F}}' \big]_{\bar{p} \bar{\alpha}} 
    &= \elm{\tilde{\Psi}}{ [H, \creaprim{\bar{\alpha}}] \anni{\bar{p}}  }{\Psi}.  \label{eq:F_check_prime_correspondence}
\end{align}
Here, the operators $\creaprim{\bar{\alpha}}$ and $\anniprim{\bar{\alpha}}$ create and annihilate
the primitive basis. We note that
\begin{align}
    \{ \anniprim{\bar{\alpha}}, \crea{p} \} &= C_{\bar{\alpha} p}, \\
    \{ \anniprim{\bar{\alpha}}, \anni{p} \} &= 0, \\
    \{ \anni{p}, \creaprim{\bar{\alpha}} \} &= \tilde{C}_{p \bar{\alpha}}, \\
    \{ \crea{p}, \creaprim{\bar{\alpha}} \} &= 0,
\end{align}
which implies the following commutators:
\begin{align}
    [\anniprim{\bar{\alpha}}, \crea{p} \anni{q}] 
    &= \{ \anniprim{\bar{\alpha}}, \crea{p} \} \anni{q} - \crea{p} \{ \anniprim{\bar{\alpha}}, \anni{q} \} \nn
    &= C_{\bar{\alpha} p} \anni{q}
\end{align}
%
\begin{align}
    [\crea{p} \anni{q}, \creaprim{\bar{\alpha}}] 
    &= - \{ \crea{p}, \creaprim{\bar{\alpha}} \} \anni{q} + \crea{p} \{ \anni{q}, \creaprim{\bar{\alpha}} \} \nn
    &= \crea{p} \tilde{C}_{q \bar{\alpha}} 
\end{align}
%
\begin{align}
    [\anniprim{\bar{\alpha}}, \crea{p} \crea{q} \anni{s} \anni{r}] 
    &= \{ \anniprim{\bar{\alpha}}, \crea{p} \} \crea{q} \anni{s} \anni{r} 
     - \crea{p} \{ \anniprim{\bar{\alpha}}, \crea{q} \} \anni{s} \anni{r}
     + \crea{p} \crea{q} \{ \anniprim{\bar{\alpha}}, \anni{s} \} \anni{r}
     - \crea{p} \crea{q} \anni{s} \{ \anniprim{\bar{\alpha}}, \anni{r} \}  \nn
    &= C_{\bar{\alpha} p} \crea{q} \anni{s} \anni{r} - C_{\bar{\alpha} q} \crea{p} \anni{s} \anni{r}
\end{align}
%
\begin{align}
    [\crea{p} \crea{q} \anni{s} \anni{r}, \creaprim{\bar{\alpha}} ] 
    &= 
    - \{ \crea{p}, \creaprim{\bar{\alpha}} \} \crea{q} \anni{s} \anni{r} 
    + \crea{p} \{ \crea{q}, \creaprim{\bar{\alpha}} \} \anni{s} \anni{r}
    - \crea{p} \crea{q} \{ \anni{s}, \creaprim{\bar{\alpha}} \} \anni{r}
    + \crea{p} \crea{q} \anni{s} \{ \anni{r}, \creaprim{\bar{\alpha}} \}  \nn
    &= -\crea{p} \crea{q} \anni{r} \tilde{C}_{s \bar{\alpha}} + \crea{p} \crea{q} \anni{s} \tilde{C}_{r \bar{\alpha}} 
\end{align}
The right-hand side of Eq.~\eqref{eq:F_check_correspondence} now becomes
\begin{align}
    \elm{\tilde{\Psi}}{ \crea{\bar{p}} [\anniprim{\bar{\alpha}}, H] }{\Psi} 
    &= \sum_{pq} h_{pq} \elm{\tilde{\Psi}}{ \crea{\bar{p}} [\anniprim{\bar{\alpha}}, \crea{p} \anni{q}] }{\Psi}
    + \frac{1}{2} \sum_{pqrs} u_{pqrs} \elm{\tilde{\Psi}}{ \crea{\bar{p}} [\anniprim{\bar{\alpha}}, \crea{p} \crea{q} \anni{s} \anni{r}] }{\Psi} \nn
    &= 
    \begin{multlined}[t]
        \sum_{pq} C_{\bar{\alpha} p} h_{pq} \elm{\tilde{\Psi}}{ \crea{\bar{p}} \anni{q} }{\Psi} \\
        + \frac{1}{2} \sum_{pqrs} \left( 
            C_{\bar{\alpha} p} u_{pqrs} \elm{\tilde{\Psi}}{ \crea{\bar{p}} \crea{q} \anni{s} \anni{r} }{\Psi} -
            C_{\bar{\alpha} q} u_{pqrs} \elm{\tilde{\Psi}}{ \crea{\bar{p}} \crea{p} \anni{s} \anni{r} }{\Psi}
        \right)
    \end{multlined} \nn
    &=  \sum_{pq} C_{\bar{\alpha} p} h_{pq} \rho_{q \bar{p}}
    + \frac{1}{2} \sum_{pqrs} \left( 
        C_{\bar{\alpha} p} u_{pqrs} \rho_{rs \bar{p} q} -
        C_{\bar{\alpha} q} u_{pqrs} \rho_{rs \bar{p} p}
    \right) \label{eq:F_check_correspondence_proof_1}
\end{align}
The last term can be simplified by renaming summation indices ($p \leftrightarrow q$ and $r \leftrightarrow s$)
followed by the identities $u_{qpsr} = u_{pqrs}$ and
$\rho_{sr\bar{p}q} = -\rho_{rs\bar{p}q}$:
\begin{align}
       \sum_{pqrs} C_{\bar{\alpha} q} u_{pqrs} \rho_{rs \bar{p} p} 
    =  \sum_{pqrs} C_{\bar{\alpha} p} u_{qpsr} \rho_{sr \bar{p} q}
    = -\sum_{pqrs} C_{\bar{\alpha} p} u_{pqrs} \rho_{rs \bar{p} q}.
\end{align}
Combining this with Eq.~\eqref{eq:F_check_correspondence_proof_1} now yields
\begin{align}
    \elm{\tilde{\Psi}}{ \crea{\bar{p}} [\anniprim{\bar{\alpha}}, H] }{\Psi} 
    &= \sum_{pq}   C_{\bar{\alpha} p} h_{pq}   \rho_{q \bar{p}}
    +  \sum_{pqrs} C_{\bar{\alpha} p} u_{pqrs} \rho_{rs \bar{p} q}  \nn
    &= \sum_{q}    h_{\bar{\alpha} q}   \rho_{q \bar{p}}
    +  \sum_{qrs}  u_{\bar{\alpha} qrs} \rho_{rs \bar{p} q}
    \label{eq:F_check_correspondence_proof_2}
\end{align}
which agrees with Eqs.~\eqref{eq:F_check_from_u_and_rho} and \eqref{eq:F_check_correspondence}.
In the last step we have used
\begin{align}
    \sum_{p}  C_{\bar{\alpha} p} h_{pq} 
    = \sum_{p \beta} C_{\bar{\alpha} p} \tilde{C}_{p \beta} h_{\beta q}
    = h_{\bar{\alpha} q}.
\end{align}
This holds if
\begin{align} \label{eq:C_C_tilde_inverse}
    \sum_{p \beta} C_{\bar{\alpha} p} \tilde{C}_{p \beta} = \delta_{\bar{\alpha} \beta},
\end{align}
which is \textit{not} true if the $p$ summation runs over active (occupied and virtual) orbitals. However,
since the primitive basis is finite,
we are free to temporarily introduce the secondary basis (i.e. the complement of the active basis) explicitly.
This means that $\mathbf{C}$ and $\tilde{\mathbf{C}}$ become square (rather than rectangular) matrices:
\begin{alignat}{2}
    \mathbf{C} 
    &= 
    \left[
	\begin{array}{c}
        \mathbf{C}_{\textsc{A}}
	\end{array} \right]
    %
    {} &&\rightarrow
    \left[
	\begin{array}{c | c}
        \mathbf{C}_{\textsc{A}} {\,} & {\,} \mathbf{C}_{\textsc{S}} 
	\end{array} \right] \\
    %%%%
    \tilde{\mathbf{C}} 
    &= 
    \left[
	\begin{array}{c}
        \tilde{\mathbf{C}}_{\textsc{A}}
	\end{array} \right]
    %
    {} &&\rightarrow
    \left[
	\begin{array}{c}
        \tilde{\mathbf{C}}_{\textsc{A}} \\
        \hline
        \tilde{\mathbf{C}}_{\textsc{S}} 
	\end{array} \right]
\end{alignat}
Since the full matrices are square we get that biorthogonality ($\tilde{\mathbf{C}} \mathbf{C} = \mathbf{1}$)
implies $\mathbf{C}  \tilde{\mathbf{C}} = \mathbf{1}$, which is exactly Eq.~\eqref{eq:C_C_tilde_inverse}.
We will never actually construct the secondary basis; we only need its \textit{existence} to complete the proof.
A similar derivation shows that
\begin{align}
    \elm{\tilde{\Psi}}{ [H, \creaprim{\bar{\alpha}}] \anni{\bar{p}}  }{\Psi}
    = \sum_{p} \rho_{\bar{p}p} h_{p \bar{\alpha}} + \sum_{pqs} \rho_{\bar{p}spq} u_{pq\bar{\alpha}s},
\end{align}
which agrees with Eqs.~\eqref{eq:F_check_prime_from_u_and_rho} and \eqref{eq:F_check_prime_correspondence}.

\begin{itemize}
    \item Orthogonal case?
    \item Ground state in exponential parameterization?
    \item Detailed derivation (also vibrational case).
\end{itemize}

\newpage
\bibliography{bib/madsgh.bib}

\end{document}
